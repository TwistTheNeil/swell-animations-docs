\subsection{Non-Functional Requirements}
The developers of this software tool are focusing on maximizing the following non-functional requirements so that users can have a better experience:

\subsubsection{Performance}
The tool performs in such a way that animated paths based on user input are created and displayed to the user in real-time. LOAs drawn by the user The animation is generated in no more than 10 seconds. 

\subsubsection{Extensibility}
The software application is built in such a way that it allows for easy modifications and expansions. Anticipated changes include modifying the user interface to support multi-path animations instead of single path animations. This would allow the user to animate two characters along the same (or different paths) on the user interface to be simulated in their game environment.

\subsubsection{Maintainability}
The application source code is divided into two seperate repositories. One repository contains code for the graphical user interface while the other repository contains code for the back end. The purpose of two separate repositories is to keep front and back end code loosely coupled and each repository will be highly cohesive. Should the method of processing the user input ever change, the development team can simply "swap out" the old method with the new method.

\subsubsection{Usability}
This tool is purposely built so that the user does not need to have extensive knowledge regarding the animation process. All animation back end computation will be taken care of for the user. Simply, the user must be able to interact with the user interface and draw a line so that an entity can animate over their path of choice. Output displayed to the user should be expected according to their input, meaning that the generated animation should be within some (user specified) parameter of the drawn LOA. This is handled in terms of constraints describing how close the user wants the generated animation to fit the drawn LOA.


\subsubsection{Portability}
This tool is made in a way that the core functions of the product are isolated within its own package. This way, the application may be ported and used on other platforms with ease. Only new UIs will have to be constructed in order to keep core functionality in other platforms. Future development may include expansion to Autodesk Maya and Unreal Engine, in addition to Unity.
