\section{Non-Functional Requirements}
The developers of this software tool are focusing on maximizing the following non-functional requirements so that users can have a better experience:

\begin{itemize}
 \item Performance
 \item Extensibility
 \item Maintainability
 \item Usability
\end{itemize}

\subsection{Performance}
The tool will perform in such a way that animated paths based on the user input should be created and displayed to the user almost immediately. Also, no unexpected output should be displayed to the user- the path that the user draws should be exactly the path that the animation follows.

\subsection{Extensibility}
The software application will be built in such a way that it will allow for easy modifications and expansions. Anticipated changes may include modifying the user interface to support multi-path animations instead of single path animations. This would allow the user to animate two characters along the same (or different paths) on the user interface to be simulated in their game environment.

\subsection{Maintainability}
The application will be coded modularly to promote polymorphic behavior and certain software design patterns. Should the method of processing the user input ever change, the development team can simply "swap out" the old method with the new method. This is possible due to minimal coupling within the code so that changing different aspects of the program will not affect its overall behavior.

\subsection{Usability}
This tool will be built for a user-friendly environment where the user need not extensive knowledge regarding animation programming. Everything on the animation end will be taken care of for the user. All the user must be able to do is interact with the user interface and draw a line so that an entity can animate over their path of choice.

\subsection{Portability}
This tool will be made in a way that the core functions of the product will be isolated within its own package. This way, the application may be ported and used on other platforms with ease.
