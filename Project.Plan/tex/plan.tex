\documentclass[12pt]{article}
\usepackage{amsmath}
\usepackage{enumitem}
\usepackage{float}

\begin{document}

\title{Pathfinding Project Plan}
\author{Sarah Kushner \\
		Neil Castelino \\
		Angel Delgado \\
		Carlo Rosati \\
		Chris Khedoo}
\date{\today}
\maketitle

\pagebreak

\section{Revision History}
\begin{table}[hp]
\centering
\begin{tabular}{|l|l|l|l|}
\hline
Name 		& Date	 		& 	Comment	 								& 	Version	 \\ \hline
All		& 10/12/15		& 	Initial project plan document			&  	1.0 		 \\ \hline
\end{tabular}
\end{table}

\pagebreak

\section{Background}
\label{sec:back}
A Unity plug-in that allows users to automatically animate ``characters" under a set of constraints by drawing a rough path from the character to a desired destination.

\subsection{Features}
(See more in the Customer Requirements Document)
\begin{itemize}
  \item Ability for user to draw a line in 3D space from a selected object to its destination
  \item The hand-drawn line to give a rough idea of the motion 
  \item Path planning techniques to calculate paths that matched the lines (as close as possible) 
  	\begin{itemize}
	  \item Genetic programming
	  \item Constraint-based optimization
	  \item Possibly look into a third option, reinforcement learning
	\end{itemize}
  \item Path should also satisfy physical constraints
	\begin{itemize}
	  \item Size and shape of character
	  \item Obstacles in the scene e.g. walls, water
	  \item Preferences of character (we will need some metric to narrow down multiple viable paths)
	\end{itemize}
  \item Present this functionality in the form of a plug-in to a game engine or animation software
\end{itemize}


\section{Statement of Work}
\label{sec:work}
\subsection{Unity UI prototype} 
A prototype will be produced in order to ensure the viability of this project.

\subsection{Infrastructure setup} 
This includes setting up the repository, automated build setup, ticket software, and any other tools needed for this project.

\subsection{UI Specifications} 
Specifications relating to the Unity UI and how the user will use to engage with the core functionality.  

\subsection{Core Functionality Specifications} 
Specifications relating to the core features of the project.

\subsection{UI Test Plan} 
The plan on how the UI will be tested and what criteria will correlate to successful implementation.  The final phase of this testing will be User acceptance testing.

\subsection{Core Functionality Test Plan} 
This plan will contain unit tests and any testing suites that need to be used in order to ensure all core functionality requirements are implemented.

\subsection{Weekly Defect Reports} 
During development a weekly report will be compiled that will include new defects and progress on existing defects. This is meant to be a tool to assist in timely defect resolution.  Solutions to roadblocks and action items relating to defects can be addressed here.

\subsection{UI Development} 
Development of the UI in Unity.

\subsection{Core Functionality Development} 
Development of the core functionality. Break project development down into smaller pieces of the whole. Develop and test each part independently, working toward finish from the ground up. Project can be stubbed out first with dummy methods.

\subsection{Product Manual} 
A manual will need to be produced to assist users and explain the functionality of the product.


\section{Resource List}
\label{sec:resource}
\subsection{Hardware}
\begin{itemize}
  \item Proper hardware to run Unity with best results
\end{itemize}

\subsection{Software}
\begin{itemize}
  \item Proper OS to develop Unity plug-ins (Windows or OSX)
  \item Student version of Unity
  \item Student version of Autodesk Maya
  \item Student version of Unreal Engine
\end{itemize}

\section{Assumptions}
\label{sec:assumptions}
Users should have general knowledge of how Unity works and is operated.

\section{Schedule}
\label{sec:schedule}
Weeks marked with (F) denote finals week, and weeks marked with (H) denote holidays.

\subsection{Term 1}
Fall 2015 \\
Goals:
\begin{itemize}
	\item Define Project Plan
	\item Define Customer Requirements
	\item Define Requirements Specification
	\item Begin Design Specification
	\item Begin Preliminary Development with core functionality for a prototype (This way, we can hopefully spot some expected problems noted in Section \ref{sec:risks} early on in the process.)
\end{itemize}

\begin{table}[H]
\centering
\begin{tabular}{|l|l|l|l|}
\hline
Week 	& Date	 		& Work	 		& 	Done?	 \\ \hline
1		& 09/21/15	 	& Form group		& 	yes		 \\ \hline
2		& 09/28/15	 	& Solidify idea	& 	yes		 \\ \hline
3		& 10/05/15	 	& Preliminary Plan document	& 	yes		 \\ \hline
4 (H)	& 10/12/15	 	& Plan document	& 			 \\ \hline
5		& 10/19/15	 	& Customer requirements document		& 			 \\ \hline
6		& 10/26/15	 	& Customer requirements document		& 			 \\ \hline
7		& 11/02/15	 	& Requirements specification	& 			 \\ \hline
8		& 11/09/15	 	& Requirements specification and Design		& 			 \\ \hline
9		& 11/16/15	 	& Design specification		& 			 \\ \hline
10 (H)	& 11/23/15	 	& Design specification and Beginning dev		& 			 \\ \hline
11		& 11/30/15	 	& Prototype development	& 			 \\ \hline
12 (F)	& 12/07/15	 	& Finalize Requirements		& 			 \\ \hline
\end{tabular}
\end{table}

\subsection{Term 2}
Winter 2016 \\
Goals:
\begin{itemize}
	\item Finalize Design Specification
	\item Continue Prototype Development with core functionality, Unity is a priority
	\item Iteratively change Requirements and Design as needed based on Protype
	\item Tighten up development and determine how we can extend our product in the time remaining
	\item Release versions of product with limited functionality
	\item Begin Testing on completed features
\end{itemize}

\begin{table}[H]
\centering
\begin{tabular}{|l|l|l|l|}
\hline
Week 	& Date	 		& Work	 		& 	Done?	 \\ \hline
1		& 01/04/16	 	& 		& 			 \\ \hline
2		& 01/11/16	 	& 		& 			 \\ \hline
3 (H)	& 01/18/16	 	& 		& 			 \\ \hline
4		& 01/25/16	 	& 		& 			 \\ \hline
5		& 02/01/16	 	& 		& 			 \\ \hline
6		& 02/08/16	 	& 		& 			 \\ \hline
7		& 02/15/16	 	& 		& 			 \\ \hline
8		& 02/22/16	 	& 		& 			 \\ \hline
9		& 02/29/16	 	& 		& 			 \\ \hline
10		& 03/07/16	 	& 		& 			 \\ \hline
11 (F)	& 03/14/16	 	& 		& 			 \\ \hline
\end{tabular}
\end{table}

\subsection{Term 3}
Spring 2016 \\
Goals:
\begin{itemize}
	\item Continue and finalize Development
	\item Develop extensions for Maya and Unreal, if time allows
	\item Continue and complete Testing on all features
	\item Prepare final release
\end{itemize}

\begin{table}[H]
\centering
\begin{tabular}{|l|l|l|l|}
\hline
Week 	& Date	 		& Work	 		& 	Done?	 \\ \hline
1		& 03/28/16	 	& 		& 			 \\ \hline
2		& 04/04/16	 	& 		& 			 \\ \hline
3		& 04/11/16	 	& 		& 			 \\ \hline
4		& 04/18/16	 	& 		& 			 \\ \hline
5		& 04/25/16	 	& 		& 			 \\ \hline
6		& 05/02/16	 	& 		& 			 \\ \hline
7		& 05/09/16	 	& 		& 			 \\ \hline
8		& 05/16/16	 	& 		& 			 \\ \hline
9		& 05/23/16	 	& 		& 			 \\ \hline
10 (H)	& 05/30/16	 	& 		& 			 \\ \hline
11 (F)	& 06/06/16	 	& 		& 			 \\ \hline
\end{tabular}
\end{table}

\section{Risks}
\label{sec:risks}
\subsection{Schedule}
\begin{itemize}
  \item One or more parts of the project may not work as expected due to its complexity
  \item We may have delays due to unforeseen events
  \item Project gets rushed and some areas of the project get neglected
  \item Team members don’t have a set time to meet with the group
\end{itemize}

\subsection{Requirements}
\begin{itemize}
  \item Requirements may change unexpectedly
  \item Requirements are not fully known at the start of the project
  \item Some requirements may be neglected as a lower priority but in reality would be a higher priority
  \item Project’s target audience, focus, or goals change unexpectedly
\end{itemize}

\subsection{Technology}
\begin{itemize}
  \item Certain team members have little to no experience with animation which may lead to delays
  \item Editor/Development suite may be buggy and lead to delays
\end{itemize}

\subsection{Management}
\begin{itemize}
  \item Project doesn’t get reviewed often and we lose track of our progress
  \item Project members haven’t yet found out the constraints for each problem they might encounter
  \item One person assumes the role of multiple management roles
\end{itemize}

\subsection{Customer}
\begin{itemize}
  \item Customer might want to change the requirements unexpectedly in a major way
  \item Customer might want a change in the schedule which the team members can’t comply with
\end{itemize}


\end{document}