\section{Challenges}
\label{sec:challenges}
We have identified what we presume will be the most challenging parts of the project. Through our prototypes, we hope to demonstrate a preliminary use case: The user will be able to draw a 2D line that a 3D snake model will follow.

\subsection{3D Line Drawing}
The line drawing will be handled by the user interface. At first, we want to allow the user to draw a 2D line using the methods described in the statement of work. Then we will move on to 3D lines, which will include experimenting with ways the user can input the path information intuitively while maintaining precise functionality.
 
\subsection{General Algorithm}
In order to break down this task, we will first employ a simplified 2D version of it to very basic models (i.e. a snake). The path-following is the main goal. The secondary goal is to have the snake model itself bend (squash and stretch) according to the path and its physical constraints. Once this is working, we can generalize it to work with 3D lines in space and with more accuracy in terms of physical animation.

\subsection{Layering of Animations}
This will be tackled in the Winter term, since the previous two points must come before this in development. Once we have the line drawing and the algorithm working, we can begin work on blending animations by means of animation layers.

