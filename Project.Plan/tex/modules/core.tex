\subsection{Core Functionality Specifications}
We will need to define the specifications relating to the core features of the project. As described later in the UI Development section, we have an initial idea of what we will need to do to implement our idea. After the prototype has been completed, we can go more into detail using the interface features and preliminary development work to show us what can, cannot, and still needs to be done. From there, we can make changes accordingly to the specifications. For example, we may find out through prototyping that the way we imagine the user drawing lines can't be done in Unity. Obviously we would have to modify that feature to fit both what we want and what is possible.

\subsection{Core Functionality Test Plan}
This plan will contain unit tests that need to be used in order to ensure all core functionality requirements are implemented correctly.

\subsection{Core Functionality Development}
Development of the core functionality. Break project development down into smaller pieces of the whole: UI frame (toolbars, tools, modes, buttons), UI functionality (path drawing, refining, copying), and core functionality (the animation generation, math). Develop and test each part independently, working toward finish from the ground up. Project can be stubbed out first with dummy methods.
